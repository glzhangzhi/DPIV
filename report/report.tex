\documentclass[UTF8]{ctexart}

% 使用CteX宏集,用于中文字符排版
\usepackage{amsmath}
% 编写公式所用宏包
\usepackage{graphicx}
% 插入图片所用宏包
\usepackage{geometry}
% 用于设置页边距
\usepackage{fancyhdr}
% 用于设置页眉页脚
\usepackage{setspace}
% 用于设置行距

\setlength{\headheight}{12.64723pt}

\title{Use Deep Learning in PIV}  % 设置标题
\author{Zhichao Zhang}  % 设置作者
\date{01.05.2023}  % 设置日期

\geometry{papersize={20cm,15cm}}  % 设置纸张大小
\geometry{left=1cm,right=2cm,top=3cm,bottom=4cm}  % 设置各个方向页边距

\pagestyle{fancy}
\lhead{Use Deep Learning in PIVr}  % 页眉区
\chead{}
\rhead{Zhichao Zhang}
\lfoot{}  % 页脚区
\cfoot{}
\rfoot{\thepage}
\renewcommand{\headrulewidth}{0.4pt}  % 页眉页脚分割线
\renewcommand{\headwidth}{\textwidth}
\renewcommand{\footrulewidth}{0pt}

\onehalfspacing  % 设置行距为字号的1.5倍,注意区分这个并不是1.5倍行距

\addtolength{\parskip}{.4em}  % 更改段间距为0.4em

% 这里是导言区,用于设置整篇文章的格式,例如页面大小,页面页脚样式,章节标题样式等

\begin{document}

\maketitle
\tableofcontents

\section{前人珠玉}

\subsection{使用深度神经网络的非监督学习}

在文章\cite{zhang_unsupervised_2020}中,提出了使用卷积神经网络对稠密光学流场进行非监督学习的方法。这篇文章中,首先分析了传统的流场速度分析方法及其优缺点,如果使用交叉相关性,计算量比较小,但是生成的流场是空间稀疏的,其稀疏程度取决于检测窗口,而且其得到的流场位移信息需要经过后处理。而基于变分的方法,虽然能生成连续的流场速度分布,但其计算量很大,因为这个一个优化问题,需要很多次的迭代来找到最优解。在当时2020年,还没有出现使用非监督学习的方法进行流场速度分析的先例。因为获取带有速度标注信息的流场速度数据集十分困难,如果使用人工数据,则一定会与真实数据存在差别,这个差别会导致当这个方法被泛化到真实数据时,存在一定的误差。

之前有使用监督学习的FlowNetS和FlowNetC\cite{fischer_flownet_2015},都在稠密光学场图像速度分析中取得了很好的成绩,而其改进版FlowNet2\cite{meister_unflow_2017}更是干到了SOTA的成绩,并且之后也提出了更为轻量化的LiteFlowNet,其在保持分析进度不变的前提下,减少了模型参数,有望使用到实时系统之中。

他还有提到在这篇文章\cite{yu_back_2016}中,提出了使用非监督学习进行流场速度分析,并提出了photometric loss和 smoothness loss,他的文章最后使用的loss正是以此为基础,加入了一个consistency loss,用于反向流场的连续性约束。

这篇文章中使用的正向流场与反向流场的思想来源于\cite{meister_unflow_2017}这篇文章。

他在文中提到,这些网络的问题有二,第一是使用了大量的标注数据,这些数据的标注工作十分繁重,不易获得,或者使用了人工数据,与真实数据存在差别,可以预见其模型的泛化能力会收到影响。二是这些网络只能用于刚体或局部刚体(rigid or quasi-rigid)。

他使用的是来自\cite{cai_flow_2021}生成的人工数据。这里一个很奇怪的点是,明明他之前批评别人使用人工数据,容易造成模型在真实数据中泛化能力差的问题,他自己又只使用人工数据。难道\cite{cai_flow_2021}生成人工数据的方法很厉害?这个待研究。

目前的计划是复现这个实验,研究其使用的数据集,以及将其结果与davis的piv方法作比较。


\subsection{一种很厉害的人工数据生成方法?}

现在是关于这篇文章\cite{cai_dense_2019}的,读这篇文章的动力是,前一篇文章中提到,他使用的数据集是来自于这篇文章的,而且他虽然自己批评了别人使用人工数据会导致泛化性不好,但是自己仍然是使用了这篇文章生成的人工数据,所以想看看这种传说中的数据生成方法到底有多厉害。

这篇文章宣称自己是第一个在piv领域全局使用cnn的模型。

文章在intro的部分花了很大的篇幅介绍piv技术,以及卷积的概念。说piv,无论是基于交叉相关性的还是关于变分的,虽然现在应用很广,而且可靠性很高,但是要么没法生成稠密速度场,要么计算量比较大。基于dnn的虽然在训练的时候计算比较大,但是做inference的时候就很快,有在实时系统使用的潜力。

这篇文章的两个主要贡献:

一是改进了FlowNetS,全局使用了卷积层。具体是在expanding part中,去掉了计算量很大且不准的插值法,以反向卷积代替,并对应的调整了不同层的损失函数权重,且对速度和图像数据都进行了归一化。

二是提出了一种人工piv数据生成的办法。根据一个matlab中的库PIVlab中的工具particle image generator,单个粒子的图像可以用两个方向的高斯分布建模。而后,生成了不同类型的速度场。这里有个问题,就是这些速度场的图像他是如何生成的?这个问题之后还是要想办法搞清楚。与此同时,他还给出了一些现成的piv图像数据集,之后可以考虑使用这些数据集来训练。将生成的粒子放入速度场,假设粒子没有质量,能够完全跟随速度场,就可以根据粒子所在的位置预测出粒子在下一个时刻的位置,由此得到两个连续的粒子分布图像,与使用的速度场一起,作为模型的input。还做了一些data augmentation


\bibliographystyle{Unsrt}
\bibliography{report}


\end{document}